\documentclass[article,A4,11pt]{llncs}
\usepackage[utf8]{inputenc}
\usepackage{amsmath}
\usepackage{amssymb}
\usepackage{amsfonts}
\usepackage{graphicx}
\usepackage{bm}

\leftmargin=0.2cm
\oddsidemargin=1.2cm
\evensidemargin=0cm
\topmargin=0cm
\textwidth=15.5cm
\textheight=21.5cm
\pagestyle{plain}
\begin{document}

\title{Building a Knowledge Graph for Scientific Computing}
\author{}
\tocauthor{René Fritze}
\institute{}
\maketitle

\begin{center}
{{\large \underline{René Fritze}}, {\large Christian Himpe}, {\large Hendrik Kleikamp}, {\large Stephan Rave}}\\
University of Münster, Germany\\
{\tt rene.fritze@wwu.de, christian.himpe@uni-muenster.de, hendrik.kleikamp@uni-muenster.de, stephan.rave@uni-muenster.de}
\end{center}

\section*{Abstract}

The Mathematical Research Data Initiative (MaRDI [1]) is a consortium of 16 Co-Applicant
research institutions. MaRDI is one of currently 19 consortia founded through Germany's National Research Data Infrastructure (NFDI [2]).
The NFDI aims to create a permanent digital repository of knowledge as an indispensable prerequisite for new research questions, findings and innovations.

In our contribution we will report on our efforts in MaRDI's Scientific Computing task area during
the first 7 months of our 5 year project duration and our plans for the future.

Central piece of the presentation will be how we are constructing a knowledge graph for numerical algorithms.
We have created an ontology that
semantically links mathematical problems with algorithms, publications and implementations.
Publications are canonically identified by their DOI, software implementations by their
swMATH [3] identifier.
The ontology encodes possible relationships between the entities in the graph and these
connections can be explored using our web-based query frontend.
This query frontend enables non-experts to quickly gain an overview of available methods and
software for numerical problems in their scientific work.
The frontend makes variations of essentially the same algorithms easily discoverable.
It allows tracking new publications or software implementations connected to a certain
mathematical problem.

We will be inviting feedback for our plans to grow this knowledge graph into a community-driven platform
with an open, freely accessible API.

\bibliographystyle{plain}
\begin{thebibliography}{10}
\bibitem{The Mathematical Research Data Initiative}
{\sc The MaRDI4NFDI Consortium}. {The Mathematical Research Data Initiative}. https://www.mardi4nfdi.de/about/mission.

\bibitem{The German National Research Data Infrastructure}
{\sc Nationale Forschungsdateninfrastruktur (NFDI) e.V.}. {The German National Research Data Infrastructure}. https://www.nfdi.de/association/?lang=en.

\bibitem{swMATH}
{\sc FIZ Karlsruhe}. {swMATH}. https://swmath.org/.
\end{thebibliography}

\end{document}
